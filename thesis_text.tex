%%%%%%%%%%%%%%%%%%%%%%%%%%%%%%%%%%%%%%%%%%%%%%%%%%%%%%%%%
%                                          
% Bakalářská práce                 
% Jiří Holuša                          
%                                          
% Jazyk: čeština
% Kódování: UTF-8
% Použitý styl: fithesis2
%
%%%%%%%%%%%%%%%%%%%%%%%%%%%%%%%%%%%%%%%%%%%%%%%%%%%%%%%%%

%%%%%%%%%%%%%%%%%%%%%%%%%%%%%%%%%%%%%%%%%%%%%%%%%%%%%%%%%
%------ Konfigurace -------
%%%%%%%%%%%%%%%%%%%%%%%%%%%%%%%%%%%%%%%%%%%%%%%%%%%%%%%%%

%% Load document class fithesis2
%% {10pt, 11pt, 12pt}
%% {draft, final}
%% {oneside, twoside}
%% {onecolumn, twocolumn}
\documentclass[11pt,draft,oneside]{fithesis2}


%% Basic packages
\usepackage{lmodern}
\usepackage[czech]{babel}
\usepackage{cmap}
\usepackage[T1]{fontenc}

\usepackage[utf8]{inputenc}
\usepackage{graphicx}
\usepackage{verbatim} 


%% Additional packages for colors, advanced
%% formatting options, etc.
\usepackage{color}
\usepackage{microtype}
\usepackage{url}
\usepackage{cslatexquotes}
\usepackage{fancyvrb}
\usepackage[small,bf]{caption}
\usepackage[plainpages=false,pdfpagelabels,unicode]{hyperref}
\usepackage[all]{hypcap}

%% Fix long URLs in DVIs
\usepackage{ifpdf}

\ifpdf
\else
  \usepackage{breakurl}
\fi

%% Packages used to generate various lists
\usepackage{makeidx}
\makeindex

\usepackage[xindy]{glossaries}
\makeglossary

%% Use STAR and CIRCLE signs for nested
%% itemized lists
\renewcommand{\labelitemii}{$\star$}
\renewcommand{\labelitemiii}{$\circ$}

%%%%%%%%%%%%%%%%%%%%%%%%%%%%%%%%%%%%%%%%%%%%%%%%%%%%%%%%%
%------- Nastavení bakalářské práce (název, autor, atd.)  -----------
%%%%%%%%%%%%%%%%%%%%%%%%%%%%%%%%%%%%%%%%%%%%%%%%%%%%%%%%%

%% Title page information
\thesistitle{Implementace fulltextového vyhledávání v issue tracking systému}
\thesissubtitle{Bakalářská práce}
\thesisstudent{Jiří Holuša}
\thesiswoman{false} %% Important when using Slovak or Czech lang
\thesisfaculty{fi}  %% {fi, eco, law, sci, fsps, phil, ped, med, fss}
\thesislang{cs}     %% {en, sk, cs}
\thesisyear{Jaro 2014}
\thesisadvisor{Mgr. Filip Nguyen}

%% Beginning of the document
\begin{document}

%% Front page with a logo and basic thesis information
\FrontMatter
\ThesisTitlePage

%% Thesis declaration (required)
\begin{ThesisDeclaration}
  \DeclarationText
  \AdvisorName
\end{ThesisDeclaration}

%% Thanks (optional)
\begin{ThesisThanks}
TODO: poděkování
\end{ThesisThanks}

%% Abstract (required)
\begin{ThesisAbstract}
TODO: abstrakt
\end{ThesisAbstract}

%% Keywords (required)
\begin{ThesisKeyWords}
TODO: klíčová slova
\end{ThesisKeyWords}

%% Beginning of the thesis itself
\MainMatter

%% TOC (required)
\tableofcontents

%%%%%%%%%%%%%%%%%%%%%%%%%%%%%%%%%%%%%%%%%%%%%%%%%%%%%%%%%
%%%%%%%%%%%%%%%%%%%%%%%%%%%%%%%%%%%%%%%%%%%%%%%%%%%%%%%%%
%%%%%%%%%%%%%%%%%%%%%%%%%%%%%%%%%%%%%%%%%%%%%%%%%%%%%%%%%
%------- Vlastní text práce  -----------
%%%%%%%%%%%%%%%%%%%%%%%%%%%%%%%%%%%%%%%%%%%%%%%%%%%%%%%%%
%%%%%%%%%%%%%%%%%%%%%%%%%%%%%%%%%%%%%%%%%%%%%%%%%%%%%%%%%
%%%%%%%%%%%%%%%%%%%%%%%%%%%%%%%%%%%%%%%%%%%%%%%%%%%%%%%%%

\chapter{Úvod}
Úvod

\chapter{Vyhledávání}
Tato kapitola stručně popisuje způsob vyhledávání v nejčastějším datovém úložišti - relačních databázích - a uvádí jeho nedostatky. Poté se detailněji věnuje jednou z možností jejich řešení, a to fulltextovým vyhledáváním. Uvádí nezbytnou 
teorii k pochopení principů, jak fulltextové vyhledávání funguje, jeho výhody a nevýhody.

\section{Vyhledávání v relačních databázích}
Relační databáze poskytují vysoce výkonný přístup k datům a široké možnosti pro jejich správu. Díky svých schopnostem se staly nejpoužívanější technologií pro datové uložiště. 
Vyhledávat v datech lze přitom pouze dvěma způsoby: porovnání obsahu buňky a operátor LIKE.

Porovnání obsahu buňky funguje na velice jednoduchém principy úplné shody obsahu. V následujícím příkladu vidíme dotaz v jazyce SQL, který vybere právě ty záznamy z tabulky People, které mají hodnotu atributu name rovnou "Bruce Banner". 
\begin{verbatim}
SELECT * FROM People WHERE name = 'Bruce Banner'
\end{verbatim}

Nebudou tedy vybrány žádné jiné záznamy, přestože by obsah atributu name měly např. "Bruce Banners" či dokonce ani "Bruce Banner " (přebytečná mezera na konci). Výhodou tohoto řešení je efektivita a jednoduchost - jediná nutná operace je 
pouze porovnání dvou řetěžců, žádné dodatečné zpracování není potřeba. 

Trochu více sofistikovaným způsobem je operator LIKE, který umožňuje (v omezené míře) používat pattern matching - vyhledávání pomocí vzoru. Podporovány jsou tzv. zástupné symboly, jež mohou mít v tomto kontextu jiný význam než jen právě daný znak, např. symbol \% 
(procento) zastupuje libovolnou sekvenci znaků (třeba i žádnou) nebo znak . (tečka) libovolný, ale právě jeden znak. Níže vidíme příklad SQL dotazu, jenž nám vrátí všechny záznamy z tabulky People, které jejich jméno končí na "Banner".
\begin{verbatim}
SELECT * FROM People WHERE name LIKE '%Banner'
\end{verbatim}

Nyní již dokázeme tímto dotazem získat jak lidi se jménem "Bruce Banner", tak i "Richard Banner". 

\section{Problémy vyhledávání v relačních databázích}
V předchozí kapitole jsme si představili základní způsoby vyhledávání v relačních databázích. Nyní se podíváme na případy, kde nám tyto způsoby nestačí nebo si s danou situací nedokáží poradit buď vůbec, nebo pouze neefektivně.

Abychom si mohli tyto nedostatky demostrovat na příkladech, uvažujme existenci jednoduché relační databáze s následujícím schématem.
\\ TODO: obrázek

\subsection{Vyhledávání přes několik tabulek}
Představme si, že uživatel zadá do vyhledávacího políčka nějaký řetězec, na jehož základě očekává odpovídající výsledky. Vyvstává otázka, kde bychom měli zadanou frázi hledat. V našem případě
pravděpodobně v nadpisu, popisu, ve jméně a příjmení autora, tam všude by se mohly nacházet informace, které uživatel hledal. 

SQL nyní musí prohledat všechny zadané sloupce, které se však mohou nacházet v různých tabulkách, což vede ke spojování tabulek. Výsledný dotaz by mohl vypadat například takto:
\begin{verbatim}
SELECT * FROM Book book LEFT JOIN book.authors author WHERE 
book.title = ? OR book.description = ? 
OR author.firstname = ? OR author.lastname = ?
\end{verbatim}

Je vidět, že i při relativně jednoduchém požadavku (vyhledáváme pouze ve čtyřech sloupcích) je výsledný dotaz poměrně složitý. Pokud bychom chtěli uživateli dát možnost využívat komplexnější
dotazy, je otázka generování odpovídajích SQL dotazu netriviální. Navíc uvažme, že musíme spojit více tabulek, což může vést k problémům s efektivitou.

\subsection{Vyhledávání jednotlivých slov}
Jak jsme si řekli, SQL dokáže vyhledat v jednotlivých sloupcích přesně zadanou frázi. Je ovšem velice nepravděpodobné, že sloupce v databázi budou obsahovat přesně stejnou danou frázi, 
hledání jednotlivých slov by nám velice zvýšilo pravděpodobnost nálezu. SQL však žádnou takovou funkcionalitu na dělení vět neposkytuje, je tedy nutné si větu předpřipravit explicitně (tj. rozdělit na slova),
a poté spouštět vyhledávácí dotaz pro každé slovo zvlášť. Následně výsledky nějakým způsobem sloučit.

\subsection{Filtrace šumu}
Některá slova ve větách nenesou vzhledem k vyhledávání žádnou informační hodnotu, např. spojky či předložky či ještě lepším příkladem mohou být anglické neurčité členy. Taková slova se nazývají šum (noise). Dále se pak některá slova v určitém kontextu
šumem stávají, např. slovo kniha v našem internetovém knihkupectví. SQL nám opět neposkytuje žádný prostředek k řešení takového problému, je tedy nutné data externě předpřipravit a případná filtrace šumu musí být vyřešena jinak.

\subsection{Vyhledávání příbuzných slov}
Je velice žádoucí, abychom se při vyhledávání mohli zaměřit pouze na význam hledaného slova, nikoliv na jeho tvar. Nemělo by záležet na tom, zda hledáme frázi fulltextové hledání nebo fulltextových vyhledávání, význam těchto frázi
je stejný. Jinak řečeno, vyhledávání by mělo brát v potaz i slova odvozená, se stejným kořenem. Ještě pokročilejším požadavkem by mohla být možnost zaměňovat slova s jejich synonymy, např. upravit a editovat.

SQL nám opět nenabízí možnost k řešení těchto požadavků, klíčem by mohl být slovník příbuzných slov a synonym a pokusit se vyhledávat i podle něj. Takové řešení však přináší nezanedbatelné množství práce, nehledě na nutnost existence takového slovníku.

\subsection{Oprava překlepů}
Uživatel je člověk a jako člověk je to tvor omylný a dělá chyby. Vyhledávání by to mělo brát v potaz a snažit se tyto překlepy opravit či uhodnout, co měl uživatel na mysli. Když v našem internetovém knihkupectví uživatel hledá knihu "Fulltextové vyhledávání" a
omylem zadá do vyhledávacího pole "Fulltetové vyhledávání", je žádoucí, aby i přes tento překlep knihu našel.

\subsection{Relevance}
Pravděpodobně největším problémem v SQL je absence jakéhokoliv mechanismu pro určení míry shody (relevance) záznamu se zadaným dotazem. Předpokládejme, že v našem knihkupectví napsal autor "John Smith" 100 knih, jednu o fulltextovém vyhledávání a 
zbytek naprosto nesouvisející s informatikou. Dále několik dalších autorů rovněž napsalo publikace na téma fulltextového vyhledávání.

Pokud jako uživatel víme, že je autorem John Smith a kniha je o fulltextovém vyhledávání, očekáváme, že na vyhledávací dotaz "John Smith fulltextové vyhledávání" obdržíme nejdříve právě chtěnou knihu, a poté teprve knihy ostatní od našeho autora či
další knihy o fulltextovém vyhledávání, jelikož naše kniha "nejvíce" odpovídala položenému dotazu.


\section{Fulltextové vyhledávání}
V předchozí kapitole jsme si uvedli, jaké problémy má vyhledávání pouze pomocí SQL. Nyní si představíme možné řešení - fulltextové vyhledávání. 

\subsection{Úvod do fulltextového vyhledávání}

\subsection{Indexace}

\subsection{Hledání}


\chapter{Dostupné technologie}
\section{Apache Lucene}

\section{Hibernate Search}

\section{Elasticsearch}

\chapter{Implementace}
V předchozích kapitolách jsme si představili fulltextové vyhledávání a dostupné technologie pro jeho implementaci na platformě Java. 

Nyní se věnujeme skutečné implementaci fulltextového vyhledávání v systému správy požadavků eShoe. Celou implementaci lze rozdělit na několik částí: výběr technologie, indexace, vyhledávání a import dat.

\section{Výběr technologie}
Pro vlastní implementaci byla zvolena technologie Elasticsearch. 

\section{Indexace}
První z problémů, který je potřeba vyřešit, je zvládnutí procesu indexace, tedy denormalizaci entit do formátu, jež se dá přímo předat Elasticsearch serveru. Jak již víme, Elasticsearch příjímá JSON-like objekty,
tedy dvojce klíč - hodnota.
Jedním z prvních možných řešení indexace je prostá manuální tvorba indexu z entity pomocí get metod, tedy pro každou třídu je vytvořen mechanismus, který v předem daném pořadí předem dané atributy získá a vytvoří z nich index.

Nevýhoda tohoto řešení je zjevná - nulová flexibilita. Při každé úpravě entity je nutné dopsat odpovídající mechanismus, který upravený atribut denormalizuje. Navíc je toto řešení udělané přesně na míru tomuto projektu, resp. přesně danému modelu entit,
tudíž není znovupoužitelné do budoucna. Výhoda je ovšem rovněž zřejmá - jednoduchost. K naprogramování takového mechanismu není potřeba víc než základní znalost jazyka Java.

Rozhodli jsme se však ubírat jinou, sofistikovanější, cestou. Denormalizace je řešena pomocí anotací velice podobně jako v Hibernate Search. 

\section{Vyhledávání}

\section{Import dat}

\chapter{Závěr}
Závěr


%%%%%%%%%%%%%%%%%%%%%%%%%%%%%%%%%%%%%%%%%%%%%%%%%%%%%%%%%
%%%%%%%%%%%%%%%%%%%%%%%%%%%%%%%%%%%%%%%%%%%%%%%%%%%%%%%%%
%%%%%%%%%%%%%%%%%%%%%%%%%%%%%%%%%%%%%%%%%%%%%%%%%%%%%%%%%
%------- Konec vlastního textu práce  -----------
%%%%%%%%%%%%%%%%%%%%%%%%%%%%%%%%%%%%%%%%%%%%%%%%%%%%%%%%%
%%%%%%%%%%%%%%%%%%%%%%%%%%%%%%%%%%%%%%%%%%%%%%%%%%%%%%%%%
%%%%%%%%%%%%%%%%%%%%%%%%%%%%%%%%%%%%%%%%%%%%%%%%%%%%%%%%%

%% Lists of tables and figures, glossary, etc.
\printindex
%\printglossary
%\listoffigures
%\listoftables

%% Bibliography from references.bib
\begingroup
\def\tmpchapter{0}
\renewcommand{\chaptername}{}
\renewcommand{\thechapter}{}
\addtocontents{toc}{\setcounter{tocdepth}{-1}}
\chapter{Literatura}
\renewcommand{\chapter}[2]{}% for other classes

\bibliographystyle{plain}
\bibliography{references}

\addtocontents{toc}{\setcounter{tocdepth}{2}}
\endgroup

%% Additional materials
\appendix

%% End of the whole document
\end{document}
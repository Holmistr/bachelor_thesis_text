%%%%%%%%%%%%%%%%%%%%%%%%%%%%%%%%%%%%%%%%%%%%%%%%%%%%%%%%%
%                                          
% Bakalářská práce                 
% Jiří Holuša                          
%                                          
% Jazyk: čeština
% Kódování: UTF-8
% Použitý styl: fithesis2
%
%%%%%%%%%%%%%%%%%%%%%%%%%%%%%%%%%%%%%%%%%%%%%%%%%%%%%%%%%

%%%%%%%%%%%%%%%%%%%%%%%%%%%%%%%%%%%%%%%%%%%%%%%%%%%%%%%%%
%------ Konfigurace -------
%%%%%%%%%%%%%%%%%%%%%%%%%%%%%%%%%%%%%%%%%%%%%%%%%%%%%%%%%

%% Load document class fithesis2
%% {10pt, 11pt, 12pt}
%% {draft, final}
%% {oneside, twoside}
%% {onecolumn, twocolumn}
\documentclass[11pt,draft,oneside]{fithesis2}

%% Basic packages
\usepackage{lmodern}
\usepackage[czech]{babel}
\usepackage{cmap}
\usepackage[T1]{fontenc}

\usepackage[utf8]{inputenc}
\usepackage{graphicx}

%% Additional packages for colors, advanced
%% formatting options, etc.
\usepackage{color}
\usepackage{microtype}
\usepackage{url}
\usepackage{cslatexquotes}
\usepackage{fancyvrb}
\usepackage[small,bf]{caption}
\usepackage[plainpages=false,pdfpagelabels,unicode]{hyperref}
\usepackage[all]{hypcap}

%% Fix long URLs in DVIs
\usepackage{ifpdf}

\ifpdf
\else
  \usepackage{breakurl}
\fi

%% Packages used to generate various lists
\usepackage{makeidx}
\makeindex

\usepackage[xindy]{glossaries}
\makeglossary

%% Use STAR and CIRCLE signs for nested
%% itemized lists
\renewcommand{\labelitemii}{$\star$}
\renewcommand{\labelitemiii}{$\circ$}

%%%%%%%%%%%%%%%%%%%%%%%%%%%%%%%%%%%%%%%%%%%%%%%%%%%%%%%%%
%------- Nastavení bakalářské práce (název, autor, atd.)  -----------
%%%%%%%%%%%%%%%%%%%%%%%%%%%%%%%%%%%%%%%%%%%%%%%%%%%%%%%%%

%% Title page information
\thesistitle{Implementace fulltextového vyhledávání v issue tracking systému}
\thesissubtitle{Bakalářská práce}
\thesisstudent{Jiří Holuša}
\thesiswoman{false} %% Important when using Slovak or Czech lang
\thesisfaculty{fi}  %% {fi, eco, law, sci, fsps, phil, ped, med, fss}
\thesislang{cs}     %% {en, sk, cs}
\thesisyear{Jaro 2014}
\thesisadvisor{Mgr. Filip Nguyen}

%% Beginning of the document
\begin{document}

%% Front page with a logo and basic thesis information
\FrontMatter
\ThesisTitlePage

%% Thesis declaration (required)
\begin{ThesisDeclaration}
  \DeclarationText
  \AdvisorName
\end{ThesisDeclaration}

%% Thanks (optional)
\begin{ThesisThanks}
TODO: poděkování
\end{ThesisThanks}

%% Abstract (required)
\begin{ThesisAbstract}
TODO: abstrakt
\end{ThesisAbstract}

%% Keywords (required)
\begin{ThesisKeyWords}
TODO: klíčová slova
\end{ThesisKeyWords}

%% Beginning of the thesis itself
\MainMatter

%% TOC (required)
\tableofcontents

%%%%%%%%%%%%%%%%%%%%%%%%%%%%%%%%%%%%%%%%%%%%%%%%%%%%%%%%%
%%%%%%%%%%%%%%%%%%%%%%%%%%%%%%%%%%%%%%%%%%%%%%%%%%%%%%%%%
%%%%%%%%%%%%%%%%%%%%%%%%%%%%%%%%%%%%%%%%%%%%%%%%%%%%%%%%%
%------- Vlastní text práce  -----------
%%%%%%%%%%%%%%%%%%%%%%%%%%%%%%%%%%%%%%%%%%%%%%%%%%%%%%%%%
%%%%%%%%%%%%%%%%%%%%%%%%%%%%%%%%%%%%%%%%%%%%%%%%%%%%%%%%%
%%%%%%%%%%%%%%%%%%%%%%%%%%%%%%%%%%%%%%%%%%%%%%%%%%%%%%%%%

\chapter{Úvod}
Úvod

\chapter{Vyhledávání}
Tato kapitola stručně popisuje způsob vyhledávání v nejčastějším datovém úložišti - relačních databázích - a uvádí jeho nedostatky. Poté se detailněji věnuje jednou z možností jejich řešení, a to fulltextovým vyhledáváním. Uvádí nezbytnou 
teorii k pochopení principů, jak fulltextové vyhledávání funguje, jeho výhody a nevýhody.

\section{Vyhledávání v relačních databázích}
Relační databáze poskytují vysoce výkonný přístup k datům a široké možnosti pro jejich správu. Díky svých schopnostem se staly nejpoužívanější technologií pro datové uložiště. 
Vyhledávat v datech lze přitom pouze dvěma způsoby: porovnání obsahu buňky a operátor LIKE.

Porovnání obsahu buňky funguje na velice jednoduchém principy úplné shody obsahu. V následujícím příkladu vidíme dotaz v jazyce SQL, který vybere právě ty záznamy z tabulky People, které mají hodnotu atributu name rovnou "Bruce Banner". 
SELECT * FROM People WHERE name = 'Bruce Banner'

Nebudou tedy vybrány žádné jiné záznamy, přestože by obsah atributu name měly např. "Bruce Banners" či dokonce ani "Bruce Banner " (přebytečná mezera na konci). Výhodou tohoto řešení je efektivita a jednoduchost - jedinná nutná operace je 
pouze porovnání dvou řetěžců, žádné dodatečné zpracování není potřeba. 

Trochu více sofistikovaným způsobem je operator LIKE, který umožňuje (v omezené míře) používat pattern matching - vyhledávání pomocí vzoru. Podporovány jsou tzv. zástupné symboly, jež mohou mít v tomto kontextu jiný význam než jen právě daný znak, např. symbol \% 
(procento) zastupuje libovolnou sekvenci znaků (třeba i žádnou) nebo znak . (tečka) libovolný, ale právě jeden znak. Níže vidíme příklad SQL dotazu, jenž nám vrátí všechny záznamy z tabulky People, které jejich jméno končí na "Banner".
SELECT * FROM People WHERE name LIKE '%Banner'

Nyní již dokázeme tímto dotazem získat jak lidi se jménem "Bruce Banner", tak i "Richard Banner". 

\section{Problémy vyhledávání v relačních databázích}
V předchozí kapitole jsme si představili základní způsoby vyhledávání v relačních databázích. Nyní se podíváme na případy, kde nám tyto způsoby nestačí nebo si s danou situací nedokáží poradit buď vůbec, nebo pouze neefektivně.

\subsection{Vyhledávání přes několik tabulek}


\subsection{Vyhledávání jednotlivých slov}

\subsection{Filtrace šumu}

\subsection{Vyhledávání příbuzných slov}

\subsection{Oprava překlepů}

\subsection{Relevance}

\section{Fulltextové vyhledávání}



\chapter{Dostupné technologie}
\section{Apache Lucene}

\section{Hibernate Search}

\section{Elasticsearch}

\chapter{Implementace}
Implementační část

\chapter{Závěr}
Závěr


%%%%%%%%%%%%%%%%%%%%%%%%%%%%%%%%%%%%%%%%%%%%%%%%%%%%%%%%%
%%%%%%%%%%%%%%%%%%%%%%%%%%%%%%%%%%%%%%%%%%%%%%%%%%%%%%%%%
%%%%%%%%%%%%%%%%%%%%%%%%%%%%%%%%%%%%%%%%%%%%%%%%%%%%%%%%%
%------- Konec vlastního textu práce  -----------
%%%%%%%%%%%%%%%%%%%%%%%%%%%%%%%%%%%%%%%%%%%%%%%%%%%%%%%%%
%%%%%%%%%%%%%%%%%%%%%%%%%%%%%%%%%%%%%%%%%%%%%%%%%%%%%%%%%
%%%%%%%%%%%%%%%%%%%%%%%%%%%%%%%%%%%%%%%%%%%%%%%%%%%%%%%%%

%% Lists of tables and figures, glossary, etc.
\printindex
%\printglossary
%\listoffigures
%\listoftables

%% Bibliography from references.bib
\begingroup
\def\tmpchapter{0}
\renewcommand{\chaptername}{}
\renewcommand{\thechapter}{}
\addtocontents{toc}{\setcounter{tocdepth}{-1}}
\chapter{Literatura}
\renewcommand{\chapter}[2]{}% for other classes

\bibliographystyle{plain}
\bibliography{references}

\addtocontents{toc}{\setcounter{tocdepth}{2}}
\endgroup

%% Additional materials
\appendix

%% End of the whole document
\end{document}